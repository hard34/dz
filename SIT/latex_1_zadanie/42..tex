
Одному  французу подарили диван,  четыре
стула и кресло.
    
Сел француз на стул у окна, а самому хочется на диване полежать.
    
Лег француз на диван, а ему уже на кресле посидеть хочется.
    
Встал француз с  дивана и сел на кресло,
как король, а у самого  мысли  в голове  уже
такие, что на кресле"=то больно пышно.  Лучше
попроще, на стуле.
    
Пересел француз на стул у окна, да только не сидится французу на этом стуле, потому
что в окно как"=то дует.
    
Француз  пересел  на стул возле печки  и
почувствовал, что он устал.
    
Тогда  француз решил лечь на диван и отдохнуть,  но, не дойдя до дивана,  свернул в
сторону и сел на кресло.
     
"---Вот где хорошо!  "--* сказал француз, но
сейчас же прибавил:  "--- А на диване-то, пожалуй, лучше.
