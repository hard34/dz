
Когда   жена  уезжает куда"=нибудь  одна,
муж бегает по комнате и не находит себе места.

Ногти у мужа страшно  отрастают,  голова
трясется, а лицо покрывается мелкими черными
точками.

Квартиранты  утешают покинутого  мужа  и
кормят его свиным зельцем. Но  покинутый муж
теряет аппетит и преимущественно пьет пустой чай.

В это время его жена купается  в озере и
случайно задевает ногой подводную корягу. 
Из"=под коряги выплывает  щука и кусает жену за
пятку.  Жена с криком выскакивает из воды  и
бежит к дому. Навстречу жене бежит хозяйская
дочка. Жена показывает хозяйской дочке пораненную 
ногу и просит ее забинтовать.
    
Вечером жена пишет мужу письмо и подробно 
описывает свое злоключение.
    
Муж читает письмо и волнуется  до  такой
степени,  что  роняет из рук стакан с водой,
который падает на пол и разбивается.
    
Муж собирает осколки стакана и ранит ими
себе руку.
    
Забинтовав пораненный  палец, муж садится и 
пишет  жене  письмо.  Потом  выходит на
улицу, чтобы бросить письмо в почтовую кружку.
    
Но  на улице муж находит папиросную  коробку, 
а в коробке 30 000 рублей.
    
Муж  экстренно  выписывает жену обратно,
и они начинают счастливую жизнь.
