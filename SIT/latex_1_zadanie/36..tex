
Однажды один человек соскочил с трамвая,
да так  неудачно,  что попал под автомобиль.
Движение уличное остановилось, и  милиционер
принялся выяснять, как  произошло несчастье.
Шофер долго что"=то объяснял, показывая пальцами на передние колеса автомобиля. Милиционер ощупал эти колеса и записал в   книжечку
название улицы.  Вокруг  собралась  довольно
многочисленная  толпа.  Какой-то  человек  с
тусклыми глазами все время сваливался с тумбы. Какая"=то дама все оглядывалась на другую
даму, а та, в свою очередь, все оглядывалась
на первую даму. Потом толпа разошлась и уличное движение восстановилось.
    
Гражданин с тусклыми  глазами еще  долго
сваливался с тумбы, но наконец и он, отчаявшись, видно, 
утвердиться на тумбе, лег просто на тротуар. В это время какой"=то человек,
несший  стул, со всего  размаха  угодил  под
трамвай. Опять пришел милиционер, опять собралась толпа и 
остановилось  уличное  движение.  И гражданин с  тусклыми глазами  опять
начал сваливаться с тумбы.
    
Ну, а  потом  опять  все стало хорошо, и
даже Иван Семенович Карпов завернул в столовую.
