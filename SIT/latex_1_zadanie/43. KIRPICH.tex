
Господин  невысокого роста с камушком  в
глазу подошел к двери табачной лавки и остановился. Его черные лакированные туфли сияли
у каменной ступенечки,  ведущей  в  табачную
лавку. Носки туфель были  направлены  внутрь
магазина.  Еще два шага,  и господин скрылся
бы за дверью.
    
Но он почему"=то задержался, будто нарочно  для  того,  чтобы подставить голову  под
кирпич, упавший с крыши.  Господин даже снял
шляпу, обнаружив свой лысый череп,  и  таким
образом кирпич ударил господина прямо по голой голове,  проломил черепную кость и застрял в мозгу.
    
Господин не упал.  Нет, он только пошатнулся от страшного удара,  вынул из  кармана
платок, вытер им лицо\dots  и,  повернувшись к
толпе,  которая мгновенно  собралась  вокруг
этого господина, сказал:
    
"--*Не беспокойтесь, господа,  у меня была
уже  прививка.  Вы видите,  у меня в  правом
глазу торчит камушек?  Это тоже был  однажды
случай. Я уже привык к этому. Теперь мне все
трын"=трава!
    
И с этими словами господин надел шляпу и
ушел  куда"=то  в сторону, оставив  смущенную
толпу в полном недоумении.
