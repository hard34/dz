
ШАШКИН  (стоя посредине сцены):  У  меня
сбежала жена. Ну что же  тут поделаешь?  Все
равно, коли сбежала, так уж не вернешь. Надо
быть философом и мудро  воспринимать  всякое
событие. Счастлив тот,  кто обладает  мудростью. 
Вот Куров этой мудростью  не обладает,
а  я  обладаю.  Я в Публичной библиотеке два
раза  книгу читал. Очень  умно там обо  всем
написано.
    
Я всем интересуюсь, даже языками. Я знаю
по"=французски считать и знаю по"=немецки  живот. Дер маген. Вот как! 
Со мной даже художник  Козлов  дружит.  Мы  с  ним вместе пиво
пьем. А Куров что? Даже на часы  смотреть не
умеет. В пальцы сморкается, рыбу вилкой ест,
спит в сапогах, зубов не чистит\dots тьфу! Что
называется "--- мужик!  Ведь с ним  покажись  в
обществе: вышибут вон, да еще и матом покроют  "---  не ходи, мол, с мужиком, коли сам интеллигент.
    
Ко  мне не подкопаешься.  Давай графа  "---
поговорю с графом. Давай барона  "--- и с бароном 
поговорю. Сразу даже не поймешь,  кто  я
такой есть.
    
Немецкий язык, это я, верно, плохо знаю:
живот "--- дер маген.  А вот  скажут мне:  "<Дер
маген фин дель мун">,  "---  а  я уже и не знаю,
чего это такое. А Куров тот и "<дер маген"> не
знает.  И ведь с таким дурнем  убежала!  Ей,
видите ли, вон чего надо!  Меня она,  видите
ли, за мужчину не считает. "<У тебя, "---  говорит, "--- голос бабий!"> Ан и не бабий,  
а  детский у меня голос!  Тонкий, детский, а вовсе
не бабий!  Дура такая!  Чего ей Куров дался?
Художник  Козлов говорит,  что с меня садись
да картину пиши.
