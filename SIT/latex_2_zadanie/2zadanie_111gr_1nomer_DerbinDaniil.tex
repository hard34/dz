\documentclass[12pt]{extarticle}
\usepackage[T2A]{fontenc}
\usepackage[utf8]{inputenc}
\usepackage[english,russian]{babel}
\usepackage[mag=1000,a4paper,   	
    left=3cm , right=2cm, top=2cm, bottom=2cm,
	]{geometry}

\begin{document}
\thispagestyle{empty}

\begin{center}
\Huge\bf
ДАНИИЛ ХАРМС

\vspace{1.5cm}
\large\bf
ВОПРОС
\end{center}

\vspace{0.15cm}
\rm
\textit{"--* Есть ли что"=нибудь на земле, что  имело 
бы значение и  могло бы даже изменить ход
событий не только на земле,  но  и  в других
мирах?} "--- спросил я у своего учителя.

\textit{"--* Есть,} "--- ответил мне мой учитель.

\textit{"--* Что же это?} "--- спросил я.

\textit{"--* Это\dots} "--- начал мой учитель и вдруг 
замолчал.


Я стоял и напряженно  ждал его ответа. А
он молчал.
    
И я стоял и молчал.
    
И он молчал.
    
И я стоял и молчал.
    
И он молчал.
    
Мы оба стоим и молчим.
    
Хо"=ля"=ля!
    
Мы оба стоим и молчим!
    
Хэ"=лэ"=лэ!
    
Да, да, мы оба стоим и молчим!

\begin{flushright}
16"=17 июля 1937 г.
\end{flushright}

\begin{center}
\vspace{0.1cm}
\large
*~*~*

\vspace{0.3cm}
\bf
ЗАБЫЛ, КАК НАЗЫВАЕТСЯ
\end{center}

\rm

Один англичанин никак не  мог вспомнить,
как эта птица называется.

\textit{"--* Это,} "--- говорит, "--- \textit{крюкица.  Ах нет, не
крюкица, а кирюкица.  Или нет,  не кирюкица,
а курякица. Фу ты! Не курякица, а кукрикица.
Да и не кукрикица, а кирикрюкица.}
    
Хотите  я вам расскажу  рассказ  про эту
крюкицу? То есть не крюкицу, а кирюкицу. Или
нет, не кирюкицу, а курякицу. Фу ты!  Не курякицу, 
а кукрикицу. Да не  кукрикицу, а кирикрюкицу!  
Нет, опять не так!  Курикрятицу?
Нет, не курикрятицу!  Кирикрюкицу? Нет опять
не так!
    
Забыл я, как эта птица называется.  А уж
если б не забыл, то рассказал бы вам рассказ
про эту кирикуркукукрекицу.

\begin{flushright}
<\dots>
\end{flushright}

\begin{center}
\vspace{0.1cm}
\large
*~*~*
\end{center}


У одной  маленькой  девочки  начал гнить
молочный зуб.  Решили эту девочку  отвести к
зубному врачу, чтобы он выдернул ей ее молочный 
зуб.
    
Вот однажды стояла эта маленькая девочка
в редакции,  стояла  она  около шкапа и была
вся скрюченная.
    
Тогда  одна редакторша  спросила эту девочку,  
почему  она стоит вся скрюченная,  и
девочка ответила, что она  стоит так потому,
что боится рвать свой молочный зуб, так  как
должно быть, будет очень больно. А редакторша 
спрашивает:

\textit{"--* Ты очень боишься, если тебя уколют булавкой в руку?}
    
Девочка говорит:
    
\textit{"--* Нет.}
    
Редакторша  уколола девочку  булавкой  в
руку и говорит,  что  рвать  молочный зуб не
больнее  этого укола. Девочка поверила и вырвала 
свой нездоровый молочный зуб.
   
Можно  только отметить находчивость этой
редакторши.

\begin{flushright}
6 января 1937 г.
\end{flushright}

\begin{center}
\vspace{0.3cm}
\large
*~*~*

\vspace{0.3cm}
\bf
ОЧЕНЬ СТРАШНАЯ ИСТОРИЯ
\end{center}

\rm 

\begin{center}
\parbox{8.5cm}{\noindent Доедая с маслом булку, \\
Братья шли по переулку. \\
Вдруг на них из закоулка \\
Пес большой залаял гулко. 
}
\end{center}

\begin{flushright}
\vspace{0.1cm}
\parbox{11cm}{\noindent Сказал младший: <<Вот напасть! \\
Хочет он на нас напасть. \\
Чтоб в беду нам не попасть, \\
Псу мы бросим булку в пасть>>. 
}
\end{flushright}

\begin{center}
\vspace{0.1cm}
\parbox{8.5cm}{\noindent Все окончилось прекрасно. \\
Братьям сразу стало ясно, \\
Что на каждую прогулку \\
Надо брать с собою\dots булку.
}
\end{center}

\begin{flushright}
1938 г.
\end{flushright}

\begin{center}
\vspace{0.1cm}
\large
*~*~*

\vspace{0.3cm}
\bf
ПАШКВИЛЬ
\end{center}

\rm 

Знаменитый чтец Антон Исаакович Ш. "---  то
самое историческое лицо, которое выступало в
сентябре месяце 1940 года в Литейном  лектории, 
"--- любило перед своими концертами  полежать  
часок"=другой и отдохнуть.  Ляжет  оно,
бывало, на кушет и скажет:
   
\textit{"--* Буду спать,} "--- а само не спит.
    
После концертов оно любило поужинать.
    
Вот оно придет домой, рассядется за столом 
и говорит своей жене:
    
\textit{"--* А ну, голубушка, состряпай"=ка мне что"=нибудь 
из лапши.}
    
И пока жена его стряпает,  оно  сидит за
столом и книгу читает.
    
Жена его хорошенькая, в кружевном передничке, 
с сумочкой в руках,  а в сумочке кружевной 
платочек  и  ватрушечный  медальончик
лежат, жена его бегает по комнате, каблучками 
стучит, как бабочки,  а  оно  скромно  за
столом сидит, ужина дожидается.
    
Все  так складно  и прилично.  Жена  ему
что"=нибудь  приятное скажет,  а оно  головой
кивает. А жена порх к буфетику  и уже рюмочками 
там звенит.
    
\textit{"--* Налей"=ка, душечка, мне  рюмочку,} "--- говорит оно.
    
\textit{"--* Смотри, голубчик, не спейся,} "--- говорит
ему жена.
    
\textit{"--* Авось, пупочка,  не сопьюсь,} "--- говорит
оно, опрокидывая рюмочку в рот.
    
А жена  грозит ему пальчиком, а сама боком 
через двери на кухню бежит.
    
Вот в  таких  приятных тонах  весь  ужин
проходит. А потом они спать закладываются.
    
Ночью,  если им мухи не мешают, они спят
спокойно, потому что уж очень они люди хорошие!

\begin{flushright}
1940 г.
\end{flushright}

\begin{center}
\vspace{0.1cm}
\large
*~*~*

\vspace{0.3cm}
\bf
УПАДАНИЕ
\end{center}

\rm

Два  человека упали с крыши пятиэтажного
дома, новостройки. Кажется, школы.  Они съехали 
по крыше в  сидячем  положении до самой
кромки и тут начали падать.
    
Их падение раньше всех заметила Ида Марковна. 
Она стояла у окна в противоложном доме 
и сморкалась в стакан. И вдруг она увидела, 
что кто"=то с крыши противоположного дома
начинает падать. Вглядевшись,  Ида  Марковна
увидела, что это начинают падать сразу целых
двое. Совершенно растерявшись,  Ида Марковна
содрала с себя рубашкуи начала этой рубашкой
скорее  протирать запотевшее оконное стекло,
чтобы  лучше  разглядеть,  кто  там падает с
крыши.  Однако сообразив,  что, пожалуй, падающие 
могут увидеть ее голой и невесть чего
про нее подумать,  Ида Марковна отскочила от
окна за плетеный треножник, на котором стоял
горшок с цветком.

В это время падающих с крыш увидела другая 
особа, живущая в том же доме,  что и Ида
Марковна, но только двумя этажами ниже. Особу 
эту тоже звали Ида Марковна. Она, как раз
в это время, сидела с ногами на  подоконнике
и пришивала к своей туфле пуговку. Взгянув в
окно, она увидела падающих с крыши. Ида Марковна  
взвизгнула и,  вскочив с подоконника,
начала  спешно открывать окно,  чтобы  лучше
увидеть,  как  падающие с крыши  ударятся об
землю. Но окно не открывалось.  Ида Марковна
вспомнила, что  она  забила окно снизу гвоздем,  
и кинулась к печке, в которой она хранила
инструменты: четыре молотка,  долото  и
клещи.
    
Схватив клещи, Ида Марковна опять подбежала 
к окну и выдернула гвоздь.  Теперь окно
легко распахнулось.  Ида Марковна высунулась
из окна и увидела, как падающие  с  крыши со
свистом подлетали к земле.
    
На улице  собралась уже небольшая толпа.
Уже  раздавались  свистки, и к месту ожидаемого 
происшествия  не спеша подходил маленького 
роста  милиционер. Носатый дворник суетился, 
расталкивая людей и поясняя,  что падающие 
с крыши могут вдарить собравшихся  по
головам.
    
К  этому времени  уже обе  Иды Марковны,
одна  в платье, а другая голая,  высунувшись
в окно, визжали и били ногами.
   
И вот наконец,  расставив руки и выпучив
глаза, падающие с крыши ударились об землю.

Так и  мы иногда, упадая с высот достигнутых, 
ударяемся об унылую клеть нашей будущности.

\begin{flushright}
7 сентября 1940 г.
\end{flushright}

\end{document} 